\documentclass[10pt, letterpaper]{article}

% Packages:
\usepackage[
    ignoreheadfoot, % set margins without considering header and footer
    top=2 cm, % seperation between body and page edge from the top
    bottom=2 cm, % seperation between body and page edge from the bottom
    left=2 cm, % seperation between body and page edge from the left
    right=2 cm, % seperation between body and page edge from the right
    footskip=1.0 cm, % seperation between body and footer
    % showframe % for debugging 
]{geometry} % for adjusting page geometry
\usepackage{titlesec} % for customizing section titles
\usepackage{tabularx} % for making tables with fixed width columns
\usepackage{array} % tabularx requires this
\usepackage[dvipsnames]{xcolor} % for coloring text
\definecolor{primaryColor}{RGB}{0, 0, 0} % define primary color
\usepackage{enumitem} % for customizing lists
\usepackage{fontawesome5} % for using icons
\usepackage{amsmath} % for math
\usepackage[
    pdftitle={Chen Yang's CV},
    pdfauthor={Chen Yang},
    pdfcreator={LaTeX},
    colorlinks=true,
    urlcolor=primaryColor
]{hyperref} % for links, metadata and bookmarks
\usepackage[pscoord]{eso-pic} % for floating text on the page
\usepackage{calc} % for calculating lengths
\usepackage{bookmark} % for bookmarks
\usepackage{lastpage} % for getting the total number of pages
\usepackage{changepage} % for one column entries (adjustwidth environment)
\usepackage{paracol} % for two and three column entries
\usepackage{ifthen} % for conditional statements
\usepackage{needspace} % for avoiding page brake right after the section title
\usepackage{iftex} % check if engine is pdflatex, xetex or luatex

% Ensure that generate pdf is machine readable/ATS parsable:
\ifPDFTeX
    \input{glyphtounicode}
    \pdfgentounicode=1
    \usepackage[T1]{fontenc}
    \usepackage[utf8]{inputenc}
    \usepackage{lmodern}
\fi

\usepackage{charter}

% Some settings:
\raggedright
\AtBeginEnvironment{adjustwidth}{\partopsep0pt} % remove space before adjustwidth environment
\pagestyle{empty} % no header or footer
\setcounter{secnumdepth}{0} % no section numbering
\setlength{\parindent}{0pt} % no indentation
\setlength{\topskip}{0pt} % no top skip
\setlength{\columnsep}{0.15cm} % set column seperation
\pagenumbering{gobble} % no page numbering

\titleformat{\section}{\needspace{4\baselineskip}\bfseries\large}{}{0pt}{}[\vspace{1pt}\titlerule]

\titlespacing{\section}{
    % left space:
    -1pt
}{
    % top space:
    0.3 cm
}{
    % bottom space:
    0.2 cm
} % section title spacing

\renewcommand\labelitemi{$\vcenter{\hbox{\small$\bullet$}}$} % custom bullet points
\newenvironment{highlights}{
    \begin{itemize}[
        topsep=0.10 cm,
        parsep=0.10 cm,
        partopsep=0pt,
        itemsep=0pt,
        leftmargin=0 cm + 10pt
    ]
}{
    \end{itemize}
} % new environment for highlights


\newenvironment{highlightsforbulletentries}{
    \begin{itemize}[
        topsep=0.10 cm,
        parsep=0.10 cm,
        partopsep=0pt,
        itemsep=0pt,
        leftmargin=10pt
    ]
}{
    \end{itemize}
} % new environment for highlights for bullet entries

\newenvironment{onecolentry}{
    \begin{adjustwidth}{
        0 cm + 0.00001 cm
    }{
        0 cm + 0.00001 cm
    }
}{
    \end{adjustwidth}
} % new environment for one column entries

\newenvironment{twocolentry}[2][]{
    \onecolentry
    \def\secondColumn{#2}
    \setcolumnwidth{\fill, 4.5 cm}
    \begin{paracol}{2}
}{
    \switchcolumn \raggedleft \secondColumn
    \end{paracol}
    \endonecolentry
} % new environment for two column entries

\newenvironment{threecolentry}[3][]{
    \onecolentry
    \def\thirdColumn{#3}
    \setcolumnwidth{, \fill, 4.5 cm}
    \begin{paracol}{3}
    {\raggedright #2} \switchcolumn
}{
    \switchcolumn \raggedleft \thirdColumn
    \end{paracol}
    \endonecolentry
} % new environment for three column entries

\newenvironment{header}{
    \setlength{\topsep}{0pt}\par\kern\topsep\centering\linespread{1.5}
}{
    \par\kern\topsep
} % new environment for the header

\newcommand{\placelastupdatedtext}{% \placetextbox{<horizontal pos>}{<vertical pos>}{<stuff>}
  \AddToShipoutPictureFG*{% Add <stuff> to current page foreground
    \put(
        \LenToUnit{\paperwidth-2 cm-0 cm+0.05cm},
        \LenToUnit{\paperheight-1.0 cm}
    ){\vtop{{\null}\makebox[0pt][c]{
        \small\color{gray}\textit{Last updated in \today}\hspace{\widthof{Last updated in \today}}
    }}}%
  }%
}%

% save the original href command in a new command:
\let\hrefWithoutArrow\href

% new command for external links:


\begin{document}
    \newcommand{\AND}{\unskip
        \cleaders\copy\ANDbox\hskip\wd\ANDbox
        \ignorespaces
    }
    \newsavebox\ANDbox
    \sbox\ANDbox{$|$}

    \begin{header}
        \fontsize{25 pt}{25 pt}\selectfont Chen Yang

        \vspace{5 pt}

        \normalsize
        \mbox{\hrefWithoutArrow{mailto:chenyang@link.cuhk.edu.cn}{chenyang@link.cuhk.edu.cn}}%
        \kern 5.0 pt%
        \AND%
        \kern 5.0 pt%
        \mbox{\hrefWithoutArrow{tel:+86-15267711552}{(+86) 15267711552}}%
        \kern 5.0 pt%
        \AND%
        \kern 5.0 pt%
        \mbox{\hrefWithoutArrow{https://yangchen73.github.io}{yangchen73.github.io}}%

        \vspace{3 pt}

        \normalsize
        The Chinese University of Hong Kong, Shenzhen, China, 518172
    \end{header}

    \vspace{5 pt - 0.3 cm}

    \section{Education}

        \begin{twocolentry}{
            09/2022 -- Present
        }
            \textbf{The Chinese University of Hong Kong, Shenzhen}\\
            Major in Computer Engineering
        \end{twocolentry}

        \vspace{0.10 cm}
        \begin{onecolentry}
            \begin{highlights}
                \item Cumulative GPA: 3.85/4.0 (Rank: 5/268 in School of Science and Engineering)
                \item Research Interests: Robot Learning, Reinforcement Learning, Deep Learning
                \item Awards \& Honors: Creativity and Innovation Award, 2024; Academic Scholarship, 2023 \& 2024; \\ 
                Dean's List, 2023 \& 2024 \& 2025
            \end{highlights}
        \end{onecolentry}

        \vspace{0.2 cm}

        \begin{twocolentry}{
            08/2024 -- 12/2024
        }
            \textbf{University of California, Berkeley}\\
            Exchange Program
        \end{twocolentry}

        \vspace{0.10 cm}
        \begin{onecolentry}
            \begin{highlights}
                \item Cumulative GPA: 4.0/4.0
                \item Related Courses: Computer Architecture, Artificial Intelligence, Discrete Math (A+)
            \end{highlights}
        \end{onecolentry}



    
    \section{Research}

        \begin{twocolentry}{
            05/2025 -- Present
        }
            \textbf{Humanoid Robot Locomotion Control via Reinforcement Learning}\\
            Research Assistant; Supervised by Prof. Ye Zhao and Feiyang Wu\\
            LiDAR Lab, Georgia Institute of Technology
        \end{twocolentry}

        \vspace{0.10 cm}
        \begin{onecolentry}
            \begin{highlights}
                \item Aimed to design efficient reinforcement learning algorithms enabling humanoid robots to walk on complex terrains
                \item Conducted simulations using IsaacLab, optimized observation space and reward design, and designed a Learn-to-Teach training framework, achieving promising results in simulation environments
                \item Planning to deploy the policy on real hardware and conduct a series of tests in real-world environments
            \end{highlights}
        \end{onecolentry}

        \vspace{0.2 cm}

        \begin{twocolentry}{
            09/2024 -- 12/2024
        }
            \textbf{UAV Path Planning and Attitude Control}\\
            Research Assistant; Supervised by Prof. Mark M. Mueller and Ruiqi Zhang\\
            High Performance Lab, UC Berkeley
        \end{twocolentry}

        \vspace{0.10 cm}
        \begin{onecolentry}
            \begin{highlights}
                \item Aimed to minimize the impact of air on each other drones while cooperating by using reinforcement learning
                \item Simulated various relative situations of two drones using Pybullet; Realized UAV attitude stabilization even when subjected to strong air current disturbances using the PPO algorithm
            \end{highlights}
        \end{onecolentry}

        \vspace{0.2 cm}

        \begin{twocolentry}{
            09/2023 -- Present
        }
            \textbf{Smart Stop Snoring Pillow}\\
            Research Assistant; Supervised by Prof. Jian Zhu and Xuanyang Xu\\
            Soft Robotics Lab, CUHKSZ
        \end{twocolentry}

        \vspace{0.10 cm}
        \begin{onecolentry}
            \begin{highlights}
                \item Aimed to design and implement a smart pillow to achieve an anti-snoring effect by detecting the user's snoring and adjusting the pillow's height to keep the user's airway clear
                \item Achieved precise control of the balloon's altitude using Poiseuille's principle to replace the flow meter with two barometers; Built an intermediate layer using ROS, achieving efficient communication between the upper computer and the microcontroller; Implemented Snoring Recognition with Spatio-Temporal Graph Neural Networks
                \item Submitted to IEEE Transactions on Mechatronics (under review, second author)
            \end{highlights}
        \end{onecolentry}

        \vspace{0.2 cm}

        \begin{twocolentry}{
            01/2025 -- 05/2025
        }
            \textbf{Online Multi-Access Scheduling Algorithm for Integrated Space-Air-Ground Networks via Inverse Reinforcement Learning}\\
            Undergraduate Thesis; Supervised by Prof. Simon Pun\\
            Space-Air-Ground Laboratory, The Chinese University of Hong Kong, Shenzhen
        \end{twocolentry}

        \vspace{0.10 cm}
        \begin{onecolentry}
            \begin{highlights}
                \item Used Gurobi solver to generate expert trajectories from small-scale instances of offline Mixed Integer Programming (MIP) problems
                \item Designed a hybrid training architecture combining Maximum Entropy IRL with PPO, utilizing expert trajectories generated by Gurobi to optimize online decision-making and achieve load balancing for HAP (High Altitude Platform)
            \end{highlights}
        \end{onecolentry}



    
    \section{Internship and Competitions}

        \begin{twocolentry}{
            04/2024 -- 08/2024
        }
            \textbf{Shenzhen Research Institute of Big Data}\\
            Research Assistant; Supervised by Dr. Yangyang Peng and Dr. Yinjun Shen
        \end{twocolentry}

        \vspace{0.10 cm}
        \begin{onecolentry}
            \begin{highlights}
                \item Aimed to achieve efficient and accurate prediction of building loads, providing valuable information for power allocation
                \item Extracted features using Fast Fourier Transform and constructed an LSTM-T-KAN model for long-term building load forecasting
                \item Submitted to Applied Energy (under review, second author)
            \end{highlights}
        \end{onecolentry}

        \vspace{0.2 cm}

        \begin{twocolentry}{
            07/2024 -- 09/2024
        }
            \textbf{2nd Prize in the Chinese Undergraduate Physics Experiment Competition}\\
            Team Leader; Supervised by Prof. Xiaolu Zhuo, Prof. Chaorui Li, and Dr. Edward Chen
        \end{twocolentry}

        \vspace{0.10 cm}
        \begin{onecolentry}
            \begin{highlights}
                \item Proposed a real-time synchronous measurement scheme for steady and alternating weak magnetic fields in a double solenoid based on the giant magnetoresistance effect and digital lock-in amplification technology
            \end{highlights}
        \end{onecolentry}

    \section{Activities}

        \begin{twocolentry}{
            01/2024 -- 05/2024
        }
            \textbf{Teaching Assistant of Mechanics (PHY 1001)}
        \end{twocolentry}

        \vspace{0.10 cm}
        \begin{onecolentry}
            \begin{highlights}
                \item Delivered presentation to illustrate physical problems in the tutorial
                \item Solved problems for students during office hours
            \end{highlights}
        \end{onecolentry}

    \section{Skills}

        \begin{onecolentry}
            \textbf{Technologies \& Frameworks:} PyTorch, Tensorflow, Pybullet, ROS, SIMD/OpenMP, IsaacLab
        \end{onecolentry}

        \vspace{0.2 cm}

        \begin{onecolentry}
            \textbf{Programming Languages:} Python, C/C++, Matlab, RISCV, Verilog
        \end{onecolentry}

        \vspace{0.2 cm}

        \begin{onecolentry}
            \textbf{Languages:} English (Fluent), Chinese (Native)
        \end{onecolentry}

    

\end{document}

